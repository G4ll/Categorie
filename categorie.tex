\documentclass[a4paper]{article}

\makeatletter
\title{Categorie}\let\Title\@title
\author{Andrea Gallese}\let\Author\@author
\date{\today}\let\Date\@date

\usepackage{Gallo}
\usepackage[normalem]{ulem}

\newcommand{\CC}{\mathcal{C}}
\newcommand{\DD}{\mathcal{D}}
\newcommand{\EE}{\mathcal{E}}

\DeclareMathOperator{\Fun}{\mathbf{Fun}}
\DeclareMathOperator{\Set}{\mathbf{Set}}
\DeclareMathOperator{\Cat}{\mathbf{Cat}}

\begin{document}
	\newgeometry{left=5cm, right=5cm, bottom=0.1cm}
	
	\Intitola
	
	\small{È possibile pensare a queste dispense come alla categoria (o, più verosibilmente, a una sottocategoria di) \textbf{Exr} degli esercizi assegnati durante il corso di Teoria delle Categoria dell'Università di Pisa dell'anno accademico 2019/20. Un morfismo tra due esercizi $ X \to Y $ è inteso essere un modo di fare seguire il risultato di $ Y $ dal risultato di $ X $. È lasciato al lettore, come esercizio, il piacere di verificare che \textbf{Exr} è effettivamente una categoria.}
	
	\restoregeometry
	
\section{Lezione 2}

\ex{Chi è l'oggetto terminale in \textbf{Top}? in \textbf{Grp}?}{
	
L'oggetto terminale in \textbf{Top} è lo spazio $ P $ costituito dal singolo punto: da ogni spazio topologico $ X $ ho esattamente una funzione $ X \longrightarrow P $ (la costante) e questa è banalmente continua.\\

Analogamente, l'oggetto terminale in \textbf{Grp} è il gruppo $ 1 $ banale.

}

\ex{La categoria prodotto $ \mathcal{C} \times \mathcal{D} $ è il prodotto in \textbf{Cat}?}{}

\ex{Dimostrare che \textbf{Cat} è una categoria.}{}
	
\ex{Dimostrare che i funtori preservano gli isomorfismi.}{}

\ex{Caratterizzare gli isomorfismi in \textbf{Cat}.}{}

\ex{Costruire una categoria \textbf{Fun}$ (\mathcal{C}, \, \mathcal{D}) $ i cui oggetti siano i funtori dda $ \mathcal{C} $ a $ \mathcal{D} $, le cui mappe siano le trasformazioni naturali.}{}

\ex{Chi sono gli isomorfismi in \textbf{Fun}$ (\mathcal{C}, \, \mathcal{D}) $?}{}

\ex{Il funtore $ F\colon \CC \to \DD $ è un'equivalenza se e solo se è un funtore essenzialmente suriettivo e pienamente fedele.}{
$ {\Rightarrow}\colon $ cominciamo mostrando che se $ F $ è un'equivalenza, allora è un funtore essenzialemnte suriettivo e pienamente fedele. Chiamiamo $ G $ il funtore nella direzione opposta, per cui sappiamo valere $ \alpha \colon 1_\CC \simeq GF $ e $ \beta\colon 1_\DD \simeq FG $.
\begin{itemize}
	\item Essenzialmente suriettivo: preso un qualunque $ Y \in \DD $, il corrispondente $ G(Y) \in \CC $ è per costruzione tale che $ \beta_Y\colon Y \to FG(Y) $ sia un isomorfismo.
	
	\item Fedele: supponiamo per assurdo che due morfismi diversi $ f, \, g \colon X \to Y $ finiscano nello stesso morfismo $ F(f) = F(g) \colon F(X) \to F(Y) $. Dal diagramma dell'equivalenza
	\[\begin{tikzcd}[ampersand replacement = \&]
	X \rar["\alpha_X"] \dar["f"]
	\& GF(X) \dar["GF(f) = GF(g)" description]
	\& X \arrow[l, "\alpha_X" above] \dar["g"] \\
	Y \rar["\alpha_Y" below]
	\& GF(Y)
	\& Y \arrow[l, "\alpha_Y"]
	\end{tikzcd}\]
	ricaviamo che $ f = g $.
	\item Pieno: prendiamo un morfismo $ f \colon F(X)\to F(Y) $ e consideriamo il diagramma
	\[\begin{tikzcd}[ampersand replacement = \&]
	X \rar["\alpha_X"] \dar["g", dashed]
	\& GF(X) \dar["G(f)"] \\
	Y \rar["\alpha_Y" below]
	\& GF(Y)
	\end{tikzcd}\]
	completato da una e una sola $ g $. Il diagramma è però ora il diagramma dell'equivalenza, da cui dobbiamo concludere che
	\[ G(f) = GF(g), \]
	da cui deduciamo, sfruttando la fedeltà di $ G $, che $ f = F(g) $.
\end{itemize}

$ {\Leftarrow}\colon $ mostriamo ora che se $ F $ è un funtore pienamente fedele ed essenzialemtne suriettivo allora riusciamo a costruire un funtore $ G $ e delle trasformazioni naturali per cui $ \alpha \colon 1_\CC \simeq GF $ e $ \beta\colon 1_\DD \simeq FG $. \\

Costruiamo $ G $. Preso un oggetto $ Y \in \DD $, grazia all'essenziale suriettività di $ F $, possiamo scegliere un oggetto $ X \in \CC $ per cui esiste un isomorfismo $ \eta_Y\colon Y \to F(X) $; chiamiamo l'oggetto $ G(Y) $. Presa una mappa $ f\colon Y_1 \to Y_2 $, consideriamo il diagramma
\[\begin{tikzcd}[ampersand replacement = \&]
Y_1 \rar["\eta_1"] \dar["f"]
\& FG(Y_1) \dar["\tilde{f}", dashed] \\
Y_2 \rar["\eta_2" ]
\& FG(Y_2)
\end{tikzcd}\]
completato da uno e un solo morfismo $ \tilde{f} $, che per piena fedeltà di $ F $ corrisponde a un unico morfismo $ G(Y_1) \to G(Y_2) $ che chiamiamo $ G(f) $. In altre parole, $ G(f) $ è l'unico morfismo in $ \Hom(G(Y_1), \, G(Y_2)) $ tale che $ FG(f) $ completi il diagramma di sopra. \\

Mostriamo che $ G $ è un funtore. Che preservi l'identità è chiaro, infatti
\[\begin{tikzcd}[ampersand replacement = \&]
Y \rar["\eta_Y"] \dar["1_Y"]
\& FG(Y) \dar["1_{FG(Y)}"] \\
Y \rar["\eta_Y" ]
\& FG(Y)
\end{tikzcd}\]
commuta. Che preservi la composizione segue dall'attenta contemplazione del diagrammone
\[\begin{tikzcd}[ampersand replacement = \&]
X \rar["\eta_X"] \dar["f"] 
\& FG(X) \dar["FG(f)"] \arrow[dd, bend left = 70, "F(G(g)G(f)) = FG(gf)", dashed] \\
Y \rar["\eta_Y"] \dar["g"]
\& FG(Y) \dar["FG(g)"] \\
Z \rar["\eta_Z"]
\& FG(Z).
\end{tikzcd}\]
La mappa $ F(G(g)G(f))$ fa commutare il quadrato esterno per funtorialità di $ F $ ed è dunque $ FG(gf) $ per definizione. Della fedeltà di $ F $ deduciamo che
\[ G(gf) = G(g)G(f). \]

È evidente che $ FG $ è naturalmente isomorfo all'identità di $ \DD $ per costruizione, mostriamo dunque che c'è una trasfromazione naturale invertibile tra $ GF $ e $ 1_\CC $. Costruiamo questa trasformazione naturale: prendiamo $ X \in \CC $ e $ F(X) \in \DD $, associato a questo oggetto tramite la scelta iniziale c'è un elemento $ X' \in \CC $ (che ufficialmente abbiamo chiamato $ GF(X') $) con tanto di isomorfismo $ \eta\colon F(X) \to F(X') $. Per piena fedeltà possiamo sollevare $ \eta $ a una mappa $ \alpha $ che, per funtorialità di $ F $, rimane un isomorfismo:
\[ \alpha_X \colon X \to GF(X). \]
Non ci resta che mostrare che $ \alpha $ è naturale. Prendiamo un morfismo $ f \colon X_1 \to X_2  $ e ricordiamoci che $ GF(f) $ è l'unico morfismo tale che il seguente diagramma commuta
\[\begin{tikzcd}[ampersand replacement = \&]
F(X_1) \rar["\eta_1 = F(\alpha_1)"] \dar["F(f)"]
\& FGF(X_1) \dar["FGF(f)"] \\
F(X_2) \rar["\eta_2 = F(\alpha_2)" ]
\& FGF(X_2).
\end{tikzcd}\]
Da cui ricaviamo che $ F(GF(f)) ) = F(\alpha_2f\alpha_1^{-1}) $ e dunque, grazie alla fedeltà di $ f $, che
\[\begin{tikzcd}[ampersand replacement = \&]
X_1 \rar["\alpha_1"] \dar["f"]
\& GF(X_1) \dar["GF(f)"] \\
X_2 \rar["\alpha_2" ]
\& GF(X_2)
\end{tikzcd}\]
commuta.
}

\section{Lezione 3}

\ex{Sia $ F\colon \CC \to \DD $ un funtore pieno, im$ (F) $ è una sottocategoria piena di $ D $?}{
Sì! Siano $ F(X) $ e $ F(Y) $ due oggetti della sottocategoria immagine, il funtore $ F $ induce per ipotesi una funzione suriettiva tra gli insiemi $ \Hom_\CC(X, \, Y) $ e $ \Hom_\DD(F(X), \, F(Y)) $, che è quello che volevamo.
}

\ex{La composizione orizzontale è ben definita. Ovvero dato si riesce a collassare
\[\begin{tikzcd}[ampersand replacement = \&]
\CC \rar["F"{name=F}, bend left = 50] \rar["G"{name=G, below}, bend right = 50]
\& \DD \rar["H"{name = H}, bend left = 50] \rar["K"{name= K, below}, bend right = 50]
\& \EE
\arrow["\alpha", Rightarrow, from=F, to=G, shorten <=4pt,shorten >=4pt]
\arrow["\beta", Rightarrow, from=H, to=K, shorten <=4pt,shorten >=4pt]
\end{tikzcd}
\qquad\text{ collassa in }\qquad
\begin{tikzcd}[ampersand replacement = \&]
\CC \arrow[rr, "HF"{name=F}, bend left = 50] \arrow[rr, "KG"{name=G, below}, bend right = 50]
\&\& \EE.
\arrow["\beta\star\alpha", Rightarrow, from=F, to=G, shorten <=4pt,shorten >=4pt]
\end{tikzcd}\]}{

C'è fondamentalmente un unico modo di costruire una trasformazione naturale tra $ HF $ e $ KG $: per ogni elemento $ X \in \CC $ otteniamo una mappa $ \alpha_X\colon F(X) \to G(X) $ in $ \DD $, che produce il seguente diagramma in $ \EE $:
\[\begin{tikzcd}[ampersand replacement = \&]
HF(X)
  \rar["\beta_{F(X)}"]
  \dar["H(\alpha_X)" left]
  \arrow[dr, dashed, "\beta \star \alpha_X" description]
\& KF(X)
  \dar["K(\alpha_X)"] \\
HG(X)
  \rar["\beta_{G(X)}" below]
\& KG(X).
\end{tikzcd}\]
Questa è effetivamente una trasformazione naturale: infatti un morfismo $ X \to Y $ produce un cubo commutativo
\[\begin{tikzcd}[row sep=small, column sep=small, ampersand replacement = \&]
	\& HF(X) \arrow[dl, red] \arrow[rr] \arrow[dd] \arrow[ddrr, red, "\beta\star\alpha_X" near start] \&\& KF(X) \arrow[dl] \arrow[dd] \\
	HF(Y) \arrow[rr, crossing over] \arrow[dd] \arrow[ddrr, red, "\beta\star\alpha_Y"{near end, below}] \& \& KF(Y) \\
	\& HG(X) \arrow[dl] \arrow[rr] \& \& KG(X) \arrow[dl, red] \\
	HG(Y) \arrow[rr] \& \& KG(Y) \arrow[from=uu, crossing over]\\
\end{tikzcd}\]
dal quale deduciamo la commutatività del quadrato di sbieco segnato in rosso, a cui siamo interessati.
}

\ex{Dimostrare che composizione orizzontale e verticale commutano.
%\[\begin{tikzcd}[ampersand replacement = \&]
%\CC
%  \rar["F"{name=F}, bend left = 50]
%  \rar["G"{name=G}]
%  \rar["H"{name=H, below}, bend right = 50] \& 
%\DD
%  \rar["J"{name = J}, bend left = 50]
%  \rar["K"{name= K}]
%  \rar["L"{name= L, below}, bend right = 50] \&
%\EE
%\arrow["\alpha", Rightarrow, from=F, to=G, shorten <=4pt,shorten >=4pt]
%\arrow["\beta", Rightarrow, from=G, to=H, shorten <=4pt,shorten >=4pt]
%\end{tikzcd}\]
}{
\[\begin{tikzcd}[ampersand replacement = \&]
JF
  \dar["J(\alpha)"]
  \rar["\gamma_F"]
  \arrow[dr, "\gamma \star \alpha", red]
  \arrow[dd, bend right = 50, dashed, "J(\beta\alpha)" left, blue]\&
KF
  \dar["K(\alpha)"] 
  \rar["\delta_F"]\&
LF
  \dar["L(\alpha)"]\\
JG
  \dar["J(\beta)"]
  \rar["\gamma_G"]\&
KG
  \dar["K(\beta)"]
  \rar["\delta_G"]
  \arrow[dr, "\delta \star \beta", red]\&
LG
  \dar["L(\beta)"]\\
JH
  \rar["\gamma_H"]
  \arrow[rr, bend right = 30, dashed, "\delta_H\gamma_H" below, blue]\&
KH 
  \rar["\delta_H"]\&
LH
\end{tikzcd}\]
}

\ex{Sia $ F\colon \CC \to \DD $ un funtore pienamente fedele, dimostrare che:
\begin{enumerate}
	\item Se $ F(f) $ è un isomorfismo, anche $ f $ è un isomorfismo.
	\item Se ho un isomorfismo $ F(X) \to F(Y) $, ne ho uno anche tra $ X $ e $ Y $.
\end{enumerate}}{
Niente di più semplice! Sia $ F(f)\colon F(X) \to F(Y) $ un isomorfismo, $ \tilde{g}= F(g) $ il suo inverso (osservare che abbiamo usato l'ipotesi di pienezza). Per funtorialità dobbiamo avere \[ F(gf) = F(g)F(f) = 1_{F(X)}. \]
La fedeltà ci dice però che un solo elmento di $ \Hom_\CC(X, \, X) $ viene mandato in $ 1_{F(X)} \in \Hom_\DD(F(X), \, F(X)) $ e questo dev'essere, per funtorialità di $ F $, l'identità di $ X $. Analogo discorso mostra che $ fg $ è l'identità di $ Y $.
}

\ex{Trovare un controesempio al lemma di sopra lasciando cadere l'ipotesi di pienezza.}{
Andiamo sul patologico.
\[ \begin{tikzcd}[ampersand replacement = \&]
\spadesuit \rar \arrow[out=180,in=90,loop] \& \clubsuit \arrow[out=0,in=90,loop]
\end{tikzcd}
\qquad
\begin{tikzcd}[ampersand replacement = \&]
F(\spadesuit) \rar[bend right = 30] \arrow[out=180,in=90,loop] \& F(\clubsuit) \lar[bend right = 30, red] \arrow[out=0,in=90,loop]
\end{tikzcd} \]
(c'è pure un esempio più piccolo!)
}

\section{Lezione 4}

\ex{Dimostrare che c'è un isomorfismo
	\[  \Hom_\mathbf{Cat}(\CC \times \DD,\, \EE) \simeq \Hom_\mathbf{Cat}(\CC, \, \Fun(\DD, \, \EE)) \qquad\text{naturale in } \CC, \, \DD, \, \EE. \]
C'è anche un'equivalenza di categorie
\[  \Fun(\CC \times \DD,\, \EE) \simeq \Fun(\CC, \, \Fun(\DD, \, \EE)) \qquad\text{naturale in } \CC, \, \DD, \, \EE? \]}{
Mortale. L'isomorfismo altro non è che una bigezione. Costruiamo mappe nelle due direzioni e mostriamo che sono una l'inversa dell'altra.
\begin{align*}
	\mathcal{A}\colon \Hom_\mathbf{Cat}(\CC \times \DD,\, \EE) & \to \Hom_\mathbf{Cat}(\CC, \, \Fun(\DD, \, \EE))\\
	\Phi & \mapsto (X \mapsto \Phi(X\times \,\bullet\,)) 
\end{align*}
Volendo, possiamo riformulare questo funtore in termini di più funtori semplici: fissato $ X \in \CC $, consideriamo il funtore $ \varphi_X\colon D \to C \times D $ che otteniamo imponendo che sulle componenti sia da un lato l'identità e dall'altro l'elemento costante $ X $ (e che manda tutte le mappe nell'identità)
\[\begin{tikzcd}[ampersand replacement = \&]
\& D
  \arrow[dl, "\text{c}_X" above]
  \arrow[d, "\varphi_X", dashed]
  \arrow[dr, "1_D"]\& \\
C \& C\times D
  \lar \rar
\& D
\end{tikzcd}  \]
}

\ex{Verificare che un monoide $ G $ in \textbf{Ab} fornito di una moltiplicazione $ m\colon G \otimes G \to G  $ è un anello.}{}

\ex{Convincersi che una categoria è un monoide nella categoria degli $ MObOb $.}{}


\ex{Verificare che $ \Hom_\CC(X, \, \bullet\,)\colon \CC \to \Set $ è un funtore.}{
}

\ex{Sia $ G $ un gruppo e $ \mathbf{BG} $ la categoria con un solo oggetto i cui morfismi sono gli elementi di $ G $:
\[\begin{tikzcd}
\spadesuit \arrow["g \in G" right, out=0,in=90,loop].
\end{tikzcd}  \]
Ci chiediamo:
\begin{enumerate}
	\item Chi sono $ \Fun(\mathbf{BG}, \,\Set) $?
	\item C'è un solo funtore da $ \mathbf{BG} $ rappresentabile: chi è?
\end{enumerate}}{
\begin{enumerate}
	\item Un funtore $ F \colon \mathbf{BG} \to \Set $ manda l'unico elemento in un insieme $ F(\spadesuit) = X $ e i morfismi in funzioni da $ X $ in sé stesso. Poiché ogni elemento di un gruppo è invertibile e i funtori preservano gli isomorfismi, l'immagine dei morfismi sarà contenuta nel gruppo delle bigezioni $ \mathcal{S}(X) $. Ricordando che i funtori rispettano la composizione otteniamo che ogni funtore corrisponde con un omomorfismo $ F \colon G \to \mathcal{S}(X) $, ovvero un'azione di $ G $ su $ X $.
	\item Il funtore è, per definizione, $ \Hom_\mathbf{BG}(\spadesuit, \, \bullet \,) $. L'unico oggetto viene mandato nell'insieme $ \Hom_\mathbf{BG} (\spadesuit, \, \spadesuit) = G $. Ogni morfismo $ g \in G $ viene mandato in $ L_g $, la moltiplicazione a sinistra per $ g $, il che definisce, coerentemente con quanto trovato prima, un'azione di $ G $ su sé stesso.
\end{enumerate}
}

\section{Lezione 5}

\begin{theorem}[Lemma di Yoneda]
Preso un oggetto $ X \in \CC $ e un funtore $ F\colon \CC \to \Set $, esiste una bigezione (naturale in $ F $ e $ X $)
\[ \Hom_{\Fun(\CC, \, \Set)}(\Hom_\CC(X, \, \bullet\,), \, F(\,\bullet\,)) \to F(X), \]
che conosciamo esplcitamente: $ \alpha \mapsto \alpha_X(id_X). $
\end{theorem}

\ex{Mostrare la naturalità.}{
	Cominciamo dalla naturalità in $ F $: supponiamo di avere una trasformazione naturale $ \beta\colon F \Rightarrow G $. Vorremmo allora che commutasse il diagramma
	\[\begin{tikzcd}[ampersand replacement = \&]
	\Hom_{\Fun(\CC, \, \Set)}(\Hom_\CC(X, \, \bullet\,), \, F(\,\bullet\,))
	  \rar
	  \dar["\beta_*"] \&
	F(X)
	  \dar["\beta_X"] \\
	\Hom_{\Fun(\CC, \, \Set)}(\Hom_\CC(X, \, \bullet\,), \, G(\,\bullet\,))
	  \rar \&
	G(X)
	\end{tikzcd}  \]
	o, equivalentemente, che $ (\beta\circ\alpha)_X(id_X) = \beta_X(\alpha_X(id_X)) $. L'ultima uguaglianza è esattamente la defizione di composizione tra trasformazioni natruali. \\
	
	La naturalità in $ X $ è meno immediata: supponiamo di avere un morfismo $ f \colon X \to Y $. Vorremo che commutasse il diagramma
	
	\[\begin{tikzcd}[ampersand replacement = \&]
	\Hom_{\Fun(\CC, \, \Set)}(\Hom_\CC(X, \, \bullet\,), \, F(\,\bullet\,))
	\rar
	\dar["(f^*)^*"] \&
	F(X)
	\dar["Ff"] \\
	\Hom_{\Fun(\CC, \, \Set)}(\Hom_\CC(Y, \, \bullet\,), \, F(\,\bullet\,))
	\rar \&
	F(Y)
	\end{tikzcd}  \]
	o, equivalentemente, che $ ((f^*)^*(\alpha))_Y (id_Y) = Ff(\alpha_X(id_X)) $. Questa volta i passaggi non sono immediati:
	\begin{align*}
		[(f^*)^*(\alpha)]_Y (id_Y) & = (\alpha f^*)_Y (id_Y) \\
								   & = \alpha_Y (f) \\
								   & = (\alpha_Yf_* )(id_Y)) \\
								   & = Ff(\alpha_Y)(id_Y)).
	\end{align*}
	L'ultima uguaglianza segue dal diagramma commutativo della naturalità di $ \alpha $ associato alla mappa $ f\colon X \to Y $:
	\[\begin{tikzcd}[ampersand replacement = \&]
	\Hom_\CC(X, \, X)
	\rar["\alpha_X"]
	\dar["f_*"] \&
	F(X)
	\dar["Ff"] \\
	\Hom_\CC(X, \, Y)
	\rar["\alpha_Y"] \&
	F(Y).
	\end{tikzcd}  \]
}

\ex{Riformulare il lemma in modo da tramutare la bigezione in un isomorfismo. (??)}{
Sembra che l'idea sia considerare la categoria prodotto $ \CC \times \Fun(\CC, \, \Set) $ e i due bifuntori verso $ \Set $ dati da $ (X, \, F) \mapsto F(X) $ e $ (X, \, F) \mapsto \Hom_{\Fun(\CC, \, \Set)}(\Hom_\CC(X, \, \bullet\,), \, F(\,\bullet\,)) $. Il Lemma di Yoneda asserisce dunque che l'esistenza di un isomorfismo tra questi bifuntori.}

\ex{Dimostrare che dal lemma segue che il funtore $ F\colon\CC^\texttt{op} \to \Fun(C, \, \Set) $ che manda $ X \mapsto \Hom_\CC(X, \, \bullet \, ) $ è pienamente fedele.}{
Per il Lemma di Yoneda applicato alla coppia $ Y, \, F $ si ha una bigezione 
\[ \Hom_{\Fun(C, \, \Set)}(\Hom_\CC(Y, \, \bullet \, ), \, \Hom_\CC(X, \, \bullet \, )) \simeq \Hom_\CC(X, \, Y), \]
che è proprio quello che volevamo.
}

\ex{Enunciare e dimostrare la forma duale del lemma di Yoneda.}{
Preso un oggetto $ X \in \CC $ e un funtore $ F\colon \CC^\texttt{op} \to \Set $, esiste una bigezione (naturale in $ F $ e $ X $)
\[ \Hom_{\Fun(\CC^\texttt{op}, \, \Set)}(\Hom_\CC(\,\bullet\, , \, X), \, F(\,\bullet\,)) \to F(X), \]
che conosciamo esplcitamente: $ \alpha \mapsto \alpha_X(id_X). $
}

\ex{Dimostrare che se 
\[ \Hom_\CC(X, \, \bullet \, ) \simeq \Hom_\CC(Y, \, \bullet \,) \]
allora $ X \simeq Y $ in $ \CC $.}{
Segue dagli esercizi 4.21 e 2.12.
}

\section{Lezione 6}

\ex{Dimostrare che il prodotto tensore tra spazi vettoriali è commutativo, associativo e banale su $ k $ usando esclusivamente la proprietà universale.}{
Pensiamo al prodotto tensore come il rappresentante del funtore Bilin$_k (V, \, W; \, \bullet \,)\colon \mathbf{Vect}_k \to \Set $ che manda ogni spazio vettoriale $ Z $ nelle funzioni bilineari $ \varphi \colon V \times W \to Z $. \\

Per mostrare che il prodotto tensore è commutativo sarà pertanto sufficiente dimostrare che i funtori Bilin$_k (V, \, W; \, \bullet \,)$ e Bilin$_k (W, \, V; \, \bullet \,)$ sono isomorfi: è chiaro che l'isomorfismo sarà dato da $$  \varphi \mapsto \varphi^*(x,\, y) = \varphi(y, \, x).  $$

Per mostrare che su $ k $ è banale sarà sufficiente osservare che Bilin$_k (V, \, k; \, \bullet \,)$ è isomorfo a $ \Hom(V, \, \bullet \,) $, che è chiaro. \\

L'associativa è lunga. Direi che l'idea è di far vedere che Bilin$_k (V \otimes W, \, Z; \, \bullet \,)$ è isomorfo al funtore che associa a uno spazio le funzioni multilineari da $ V, \, W, \, Z $: mi basterà mandare una bilieare $ \varphi $ in una multilineare $ \psi(V, \, W, \, Z) = \varphi(\otimes(V, \, W), \, Z). $ Dove $ \otimes $ è lo speciale elemento di Bilin$_k (V, \, W ; V \otimes W)$ selezionato da Yoneda (che arriva dall'isomorfismo della rappresentabilità?).
}

\ex{Verificare che la versione superdebole della corrispondenza di Tannaka funziona: ovvero la composizione in $ \Aut{F} $ coincide con quella di $ \Hom_{G-\text{set}}(G, \, G) $.}{
Il funtore dimenticante $ F \colon G-\text{set} \to G $ è rappresentato da $ \Hom(G, \, \bullet \, ) $, da cui
\[ \text{End}_{\Fun} (F) \simeq \Hom_{}(G, \, G) = G \]
per Yoneda. Dunque presi $ \alpha, \, \beta \in \text{End}_{\Fun} (F) $ si ha che
\[ (\beta \circ \alpha)_G(id_G) = \beta_G(\alpha_G(id_G)) \]
che è la mappa che manda $ 1 \mapsto \beta_G \circ \alpha_G \circ id_G (1). $
}

\ex{Mostrare che le due definizioni di colimite coincidono.}{
Ricordiamo brevemente le due definizioni:
\begin{enumerate}
	\item L'oggetto $ X $ che rappresenta Coni$(F, \, \bullet \, ) \colon \CC \to \Set $.
	\item L'oggetto $ Y $ iniziale nella categoria dei coni sotto $ F $.
\end{enumerate}
Osserviamo che \textbf{la prima definizione è sbagliata}: infatti $ X \in \CC $ non può essere un cono. L'errore non è però solo formale, infatti non è stata proprio definita la trasformazione naturale $ F \Rightarrow X $! Per farlo abbiamo bisogno di invocare Yoneda 
\[ \Hom_{\Fun(\CC, \, \Set)}(\Hom_\CC(X, \, \bullet\,), \, \text{Coni}(F, \, \bullet \, )) \simeq \text{Coni}(F, \, X ): \]
allo specifico isomorfismo di rappresentabilità a sinistra $ \Phi\colon\Hom_\CC(X, \, \bullet \,) \simeq\text{Coni}(F, \, \bullet \, )  $, corrisponde infatti un cono $ \chi\colon F \Rightarrow X $ a destra. Equivalentemente, possiamo pensare all'oggetto limite come all'elemento $ \Phi_X(id_X) $ di $ \text{Coni}(F, \, X ) $. \\

Preso dunque un cono qualunque $ \gamma \colon F \Rightarrow Z $ sotto $ F $, questo è identificato, tramite l'isomorfismo $ \Phi_Z \colon\Hom_\CC(X, \, Z) \simeq\text{Coni}(F, \, Z)  $, da un unico elemento $ f \in \Hom_\CC(X, \, Z) $ (insomma, possiamo pensare al cono in questione come all'elemento $ \Phi_Z(f) $). Dalla naturalità di $ \Phi $ otteniamo il diagramma commutativo
\[\begin{tikzcd}[ampersand replacement = \&]
\Hom_\CC(X, \, X)
  \rar["\Phi_X" ]
  \dar["f^*" ] \&
\text{Coni}(F, \, X)
  \arrow[d, "C_f"  ]\\
\Hom_\CC(X, \, Z)
  \rar["\Phi_Z"]
\&\text{Coni}(F, \, Z),
\end{tikzcd}  \]
che dice, in particolare, che $ \gamma = \Phi_Z(f) = C_f(\Phi_X(id_X)) = C_f(\chi) $, ovvero che il cono $ \Phi_Z(f) $ è effettivamente quello che si ottiene componendo il limite $ \chi $ con $ f $ a sinistra (o meglio, con il morfismo tra coni identificato da $ f $). Viceversa, un morfismo di coni da $ \chi $ a $ \gamma $ è necessariamente della forma $ C_g $ per una qualche mappa $ g\colon X \to Z $, la quale a sua volta identifica, tramite l'isomorfismo $ \Phi_Z $ un cono $ \Phi_Z(g)=C_g(\chi) $, che può coincidere con $ \gamma $ solamente se $ g = f $ (per bigettività di $ \Phi_Z $). Abbiamo dunque mostrato che esiste un unico morfismo di coni da $ \chi $ a un qualunque cono $ \gamma $, ovvero che $ \chi $ è iniziale nella categoria dei coni sotto $ F $. \\

Il viceversa sarà più scorrevole. Chiamiamo $ \chi\colon F \Rightarrow Y $ il limite e sia, come prima, $ \gamma \colon F \Rightarrow Z $ un cono qualunque sotto $ F $; per ipotesi esiste un unico morfismo di coni da $ \chi $ a $ \gamma $ e, come osservato in precedenza, questo coincide con la composizione per una qualche morfismo $ f_\gamma\colon Y \to Z $. In particolare, possiamo scrivere $ \gamma = C_{f_\gamma} (\chi) $: questo definisce in modo univoco una trasformazione di componenti
\begin{align*}
	\Psi_Z\colon \text{Coni}(F, \, Z) &\to \Hom_\CC(Y, \, Z)\\
	\gamma & \mapsto f_\gamma,
\end{align*}
la cui naturalità segue dall'osservare che, per ogni $ f \colon Z \to W $ in $ C $, si ha $ f\Psi_Z(\gamma) = \Psi_W(C_f(\gamma)) $; infatti:
\begin{align*}
	C_{f\Psi_Z(\gamma)}(\chi) &
	= C_{f} C_{\Psi_Z(\gamma)}(\chi) & \text{ per funtorialità di } C, \\
	& = C_{f}(\gamma) & \text{ per definizione di } \Psi_Z, \\
	& = C_{\Psi_W(C_f(\gamma))}(\chi) & \text{ per definizione di } \Psi_W.
\end{align*}}

\section{Lezione 7}
\ex{Si consideri il seguente diagramma {commutativo} in $ \CC $:
\[\begin{tikzcd}[ampersand replacement = \&]
A \rar\dar\& B \rar\dar\& C \dar \\
D \rar\& E \rar\& F.
\end{tikzcd}  \]
Si dimostri che:
\begin{enumerate}
	\item se i due piccoli quadrati sono cartesiani, lo è anche quello grosso;
	\item se il quadrato grosso e \textbf{quello di destra} sono cartesiani, lo è ache quello di sinistra.
\end{enumerate}}{
1. Niente di più facile! Prendiamo un oggetto $ X $ adornato di due mappa $ X \to D $, $ X \to C $ che producono un diagramma commutativo
\[\begin{tikzcd}[ampersand replacement = \&]
X \arrow[dd, bend right = 50, blue]
  \arrow[rrd, bend left = 30, blue]
  \arrow[rd, bend left = 30, dashed, "1" description, red]
  \arrow[d, dashed, "2", red]
\&\&\\
A \rar\dar\& B \rar\dar\& C \dar \\
D \rar\& E \rar\& F.
\end{tikzcd}  \]
Componendo $ X\to D \to E $, scopriamo che $ X $ è un cono sopra la $ \lrcorner $ del quadrato di destra, la cui cartesianità produce un'unica mappa $ X \to B $. Questa mappa fornisce ad $ X $ una struttra di cono sopra la $ \lrcorner $ del quadrato di sinistra, la cui cartesianità produce un'unica mappa $ X \to A $. Questo dimostra che il quadrato grosso è cartesiano: l'unicità di questa mappa segue infatti osservando che ogni mappa $ X \to A $ con le proprietà richieste ne induce una $ X \to B $ che deve necessariamente coincidere con $ (1) $. \\

2. Questo caso è più sottile: per costruire la mappa $ X \to A $ è sufficiente che il quadrato grosso sia commutativo:
\[\begin{tikzcd}[ampersand replacement = \&]
X \arrow[dd, bend right = 50, blue]
  \arrow[rd, bend left = 30, blue]
  \arrow[rrd, bend left = 30, dashed, blue]
  \arrow[d, dashed, "1", red]
\&\&\\
A \rar\dar\& B \rar\dar\& C \dar \\
D \rar\& E \rar\& F.
\end{tikzcd}  \]
Molto sospetto.
}


\ex{Costruire gli equalizzatori usando solo la parola ``pullback''.}{
La costruzione non è completamente immediata: secondo me il modo giusto è dire che il pullback di 
\[\begin{tikzcd}[ampersand replacement = \&]
\& X \dar["1_X \times g"] \\
X \rar["1_X\times f"] \& X \times Y. 
\end{tikzcd}  \]
Questo sarà un oggetto del tipo
\[\begin{tikzcd}[ampersand replacement = \&]
E \rar["p_1"]\dar["p_2"] \& X \dar["1_X \times g"] \\
X \rar["1_X\times f"] \& X \times Y,
\end{tikzcd}  \]
universale. Questa formulazione ci permette di concludere che $ p_1 = p_2 $, infatti
\[ p_1 =  \pi_X \circ (1_X \times g) \circ p_1 = \pi_X \circ (1_X \times f) \circ p_2 = p_2.  \]
L'unico trucco è che abbiamo usato la parola ``prodotto'', di cui però non possiamo fare a meno (credo).
}

\ex{Completare la dimostrazione della completezza di $ \Set $ con le verifiche opportune.}{

}

\ex{Sia $ \CC $ una categoria in cui troviamo i morfismi $ f \colon X \to Y $ e $ g \colon Z \to W $. Costruire l'insieme dei quadrati commutativi con $ f $ e $ g $ orrizontali come pullback in $ \Set $.}{
L'abbiamo praticamente fatto in classe: DiagComm$ (f,\, g) $ è il pullback in $ \Set $ di
\[\begin{tikzcd}[ampersand replacement = \&]
\& \Hom_\CC(Y, \, W) \dar["f_*"] \\
\Hom_\CC(X, \, Z) \rar["g^*"] \& \Hom_\CC(X, \, W). 
\end{tikzcd}  \]

}

\ex{Quanto sono belli i funtori dimenticanti? Quanto sono belli i funtori liberi?}{
Non tutti i funtori dimenticanti preservano i limiti. Per dire, il prodotto in \textbf{Top}$ _* $ (la categoria degli spazi topologici puntati) non ha come insieme soggiacente il prodotto degli insiemi dei due spazi puntati. Questo ci dice anche che non può riflettere i limiti. Però forse non era proprio questa la domanda...
}

\end{document}